\section{Related Work}\label{sec:relatedwork}
% Related work
% - e-syn
% - rover
% - egraphconstraints
% - dsd
% - adaptdecomp
\todo{explain related work}

\subsection{LUT Packing}\label{sec:relatedwork:packing}

\subsection{LUT Decomposition}\label{sec:relatedwork:decomp}

\subsection{E-Graph Superoptimization}\label{sec:relatedwork:egraph}
Equivalence graphs (e-graphs for short) are a data structure that were
originally conceived to facilitate automated proof generation. For example,
e-graphs can be used to simplify mathematical expressions~\cite{egraphmath} or
for automated reasoning about functional programs~\cite{cclemma}. In recent
years, e-graphs~\cite{eggpaper} and equality saturation~\cite{eqsat} have
enjoyed renewed popularity within the compilers field. Equality saturation is
useful for optimizing compilers, because they defer greedy decisions.
Extracting solutions from e-graphs can result in more optimal---sometimes
provably optimal---program transformations. In contrast, traditional compiler
pass pipelines suffer from a \textit{phase-ordering problem}.
