% use the base acmart.cls
% use the sigplan proceeding template with the default 10 pt fonts
% nonacm option removes ACM related text in the submission. 
\documentclass[sigplan,nonacm]{acmart}
\usepackage{csvsimple}

\newcommand{\fixme}[1]{\textcolor{red}{#1}}
\newcommand{\todo}[1]{\textcolor{red}{TODO:\ #1}}
\newcommand{\fullname}{More Exact Technology-Mapping using E-Graphs}
\newcommand{\shortname}{E-Pack}
\newcommand{\B}{\mathbb{Z}_{2}}
\newcommand{\Bk}{\mathbb{Z}_{2}^{k}}
\newcommand{\Bx}[1]{\mathbb{Z}_{2}^{#1}}

% enable page numbers
\settopmatter{printfolios=true}


\begin{document}

\title{\fullname}

\begin{abstract}
    FPGA technology mapping is a well-studied problem and has been an area of
    interest in EDA tool design for decades. In most respects, the computational
    complexity of technology mapping is understood, and heuristic algorithms have
    been successfully employed to mitigate compile times while maintaining high QoR
    (quality of results). As transistor scaling comes to an end within the coming
    years, logic synthesis tools will become more of a bottleneck in the design of
    high performance accelerators. As a solution, we introduce E-Pack, an e-graph
    driven technology mapper that can better span the wide gap between SAT-based
    exact synthesis and heuristic cut enumeration techniques. We show that E-Pack
    can synthesize circuits with 5\% fewer LUTs on average---without ever
    increasing circuit depth. We also provide an empirical analysis of the runtime
    of E-Pack and show that it is still practical for large designs. Finally, we
    demonstrate that our compiler infrastructure is reusable, and future work can
    use our compiler for RTL equivalence checking or auditing the QoR of synthesis
    tools.
\end{abstract}
\maketitle % should come after the abstract

\section{Introduction}\label{sec:intro}
Given the complexity of modern electronic systems, a high degree of automation
is required to develop custom hardware within sensible timelines. At the
highest level, FPGA and ASIC design flows can be split into logical synthesis
and physical synthesis (optimize timing, placement and routing, etc..). This
division of work produces suboptimal designs, and neither are the individual
synthesis steps locally optimal on their own. However, logic minimization
problems in general are NP-Hard~\cite{logicmin,twolevellogic}, and modern EDA
flows bring compile times down to the human timescale while maintaining
acceptable QoR (quality of results).

With the end of Moore's Law scaling, chip area becomes a tighter constraint,
and logic synthesis is more of a bottleneck. Hence, future synthesis tools will
need to expand the design spaces they explore and find more optimal solutions.
Nonetheless, finding provably optimal circuits is computationally intractable.
In this paper, we will introduce how FPGA technology mapping can be augmented
with e-graph data structures to find \textit{more} exact solutions, without
significantly increasing compile times.

Technology mapping is the hand-off between logical synthesis and physical
synthesis. It converts the abstract Boolean logic into a network of gates that
belong to the target cell library. For FPGAs, the primary target cell is the
LUT (lookup table). Since every $k$-LUT can be re-programmed to satisfy any $k$
input boolean function, FPGA technology mapping has an unmistakably large
solution space. Whether the circuit is optimized for latency or area, most FPGA
tools approach technology mapping as a graph covering problem~\cite{flowmap,
    daomap, attmap, imap}. In the literature, a group of circuit nodes implemented
by a $k$-LUT is called a $k$-feasible cut of logic, and the generation of all
cuts is called cut enumeration. These structural mapping techniques
fundamentally rely on the topology of the input circuit. Hence, they are prone
to structural bias.

In contrast, functional mappers attempt to decompose Boolean functionality into
smaller sub-functions which can be realized by $k$-LUTs. Such mappers are a
more exact approach, and often use SAT solvers~\cite{satmap,satmap2} to drive
synthesis. However, exact synthesis tools cannot be scaled past tens of gates.
As a consequence, cut enumeration and functional mapping lie on two different
extremes. The former is faster but limited by the input structure, while the
latter is unbiased but fundamentally unscalable.

For this reason, we propose an e-graph driven technology mapper than can better
span the time-QoR spectrum. E-graphs are a data structure which use union-find
operations to compactly represent abstract equivalence relations. E-graphs are
useful as an optimizing compiler framework, because terms can be iteratively
rewritten in a nondestructive fashion. Instead of an optimization pass
architecture, e-graph driven compilers store all transformed terms in parallel
and defer selection of the best one. Our work seeks to evaluate the suitability
of e-graphs for logic synthesis, specifically technology mapping to FPGAs.

We introduce \shortname{}: a tool for repacking FPGA netlists into more compact
forms---without increasing circuit depth. Our results show many benchmarks, big
and small, which synthesize to significantly fewer LUTs over vendor EDA tools.
To that end, our work makes the following contributions:

\begin{itemize}
    \item We formulate an intermediate language and generating set of e-graph rewrite
          rules, as well as justify the types of circuit topologies that are reachable
          under composition.
    \item We evaluate our compiler against 96 benchmarks combined from three sources:
          EPFL~\cite{epflbench}, ISCAS'85~\cite{iscasbench}, and
          LGSynth'91~\cite{lgsynthbench}.
    \item \shortname{} is packaged as a Verilog-to-Verilog tool that can be dropped into existing RTL flows.
\end{itemize}

Before elaborating on our methodology and experimental setup, we first discuss
related ideas in technology mapping and e-graph driven compilers. Then, the
results section illustrates the typical reduction in LUT count our tool
achieves without increasing circuit depth. Lastly, we discuss the future work
of our compiler.
\section{E-Graph Construction} \label{sec:rewrites}

\subsection{Simplifying Degenerate LUTs} \label{sec:rewrites:degen}

\textbf{Definition:} A LUT's configuration $F : \{ 0, 1 \}^k \rightarrow \{ 0, 1 \}$ is \textit{degenerate} if there exists a Shannon expansion $F = x_i \cdot F_{x_i} + \overline{x_i} \cdot F_{\overline{x}_i}$
such that $F_{x_i} = F_{\overline{x}_i}$ for some $i \in \{ 0, \ldots, k -1\}$. In other words, $F = F_{x_i} = F_{\overline{x}_i}$.

The output of a degenerate LUT is not dependent on one of its inputs. Hence, it
can be rewritten into a LUT which uses fewer inputs. This rule is applied by
computing the Shannon expansions of LUTs and checking for equivalence.

\begin{verbatim}
    (LUT F x0 x1 x2) => (LUT F' x0 x1)
        if F(x0, x1, 0) == F(x0, x1, 1)
        where F'(x0, x1) := F(x0, x1, 0)
\end{verbatim}

A separate rule is instantiated for every LUT size and every input position. In
total, $\sum_{k=1}^{6} k = 6(6+1)/2 = 21$ rules are instantiated for searching
for degenerate LUTs. Since this rule is computationally expensive, it is
applied greedily as a pre-processing step before the e-graph is built. None of
the other rewrite rules create degenerate LUTs, so this has no impact on
results. Of course this rule can be enabled at any time, if it were necessary.

\subsection{Functional Composition}  \label{sec:rewrites:composition}

\subsection{LUT Symmetries} \label{sec:rewrites:symmetry}

\subsection{LUTs with Domain Restrictions} \label{sec:rewrites:restrict}

\textbf{Definition:} A lookup table \texttt{(LUT F x0 x1 ...)} is \textit{restricted} if \texttt{xi == xj} for some $ i, j \in \{0, \ldots, k-1\}, \; i \neq j$.
In other words, the domain of the LUT is restricted.

The main advantage of using e-graphs is the compact way in which it represents
notions of equality. When considering the entire set of rewrite rules under
composition, we can observe new equalities being formed in the e-graph.
Whenever an equality is found between two of the inputs to a $k$-LUT, it can be
rewritten with a $(k-1)$-LUT. We simply need to define and compute
$\texttt{restrict(F, i, j)}$ which maps $F : \{0, 1\}^k \rightarrow \{ 0, 1\}$
to the domain-restricted $F \vert_{x_i = x_j} : \{0, 1\}^{k-1} \rightarrow \{
    0, 1\}$. In pseudocode, the rewrite rule can be rwritten as follows:

\begin{verbatim}
    (LUT F x0 x1 x0) => (LUT G x0 x1)
        where G := restrict(F, 0, 2)
\end{verbatim}

Since e-graph rewrite patterns search on e-classes, this rule is automatically
re-checked when e-classes are merged. In total, there are as many domain
restriction rules as there are combinations of positions in which inputs can
alias: $\sum_{k=2}^{6} \binom{k}{2} = 35$ rules.

\subsection{Functional Decomposition} \label{sec:rewrites:decomp}

\input{sections/xx-toolflow.tex}
\section{Results}\label{sec:results}

\subsection{Benchmarks}\label{sec:results:benchmark}
\todo{30 out of 96 benchmarks improve by p percent}

\subsection{Marginal Improvement}\label{sec:results:margin}
\todo{graph showing improvement with iter count}

\subsection{Pipelined Designs}\label{sec:results:retiming}
\todo{pipelined mult design}

\subsection{Bit-Parallel Designs}\label{sec:results:scalability}
\todo{increasing ALU bit parallelism}

\subsection{Runtime Complexity}\label{sec:results:complexity}
\todo{graph showing marginal runtime cost of each iteration}


% Possible outline:
% 1. Related work
% 2. Egraph Construction
% 2.a. Simplifying degenerate LUTs
% 2.b. Functional composition
% 2.c. Functional decomposition
% 2.d. Symmetries of non-degenerate LUTs
% 2.e LUTs under restriction
% Experimental Setup
% results
% Future work

% use the ACM bibliography style
\bibliographystyle{ACM-Reference-Format}
\bibliography{references}

% \newpage
%%
%% If your work has an appendix, this is the place to put it.
% \appendix

\end{document}